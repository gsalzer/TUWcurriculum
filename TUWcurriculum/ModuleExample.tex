% TU Wien Curriculum 2025.2

% Use German or English, depending on the curriculum.
% Ask the head of the curricula commission (Studienkommission) for instructions.
\documentclass{TUWcurriculum}
\begin{Modul}
  % title of module
  \Titel{Theoretische Informatik}

  % size of module in Ects; may be a range, like "6,0--9,0"
  \Ects{6,0}

  % responsible person, only used internally
  \Redaktion{Gernot Salzer}

  % version information, e.g. the current date; only used internally
  \Version{2025-01-09}

  % short description, used standalone in the main part of the curriculum
  \begin{Kurzbeschreibung}
    Dieses Modul führt in die Kerngebiete der Theoretischen Informatik
    ein, wobei folgende Themengebiete im Mittelpunkt stehen: Automaten
    und formale Sprachen, Berechenbarkeit und Komplexität, sowie die
    Grundlagen der formalen Semantik von Programmiersprachen.
  \end{Kurzbeschreibung}

  % THE FOLLOWING SECTIONS ARE USED IN THE MODULE DESCRIPTION IN THE APPENDIX.
  
  % learning outcomes
  \begin{Lernergebnisse}
    % introductory sentences beneath the headline
    \begin{Intro}
      Dieses Modul befasst sich mit den theoretischen Grundlagen der
      Informatik.
    \end{Intro}

    % topic-specific/technical competencies taught by the module
    \begin{Fachlich}
      Fundamentale Konzepte und Resultate in folgenden Gebieten:
      Automaten und formalen Sprachen, Berechenbarkeit und Komplexität
      sowie die Grundlagen der formalen Semantik von Programmiersprachen.
    \end{Fachlich}

    % interdisciplinary competencies taught by the module
    \begin{Ueberfachlich}
      Die Studierenden erwerben die Fähigkeit, formale Beschreibungen
      lesen und verstehen und Konzepte formal-mathematisch beschreiben
      zu können. Weiters lernen sie, die Struktur von Beweisen und
      Argumentationen zu verstehen und selber solche zu führen.
      Weiters können die Studierenden ethische Fragestellungen im
      Kontext der Inhalte des Moduls identifizieren, formulieren und
      diskutieren.
    \end{Ueberfachlich}

    % Optional: Unterteilung "Überfachlich" in die folgenden Gruppen
    \begin{Methodenkompetenzen}
      Zeitmanagement, Arbeitstechniken,
      wissenschaftliches Schreiben, Umgang mit KI,
      Präsentationstechniken, Wissenschaftskommunikation
    \end{Methodenkompetenzen}
    \begin{Sozialkompetenzen}
      Teamfähigkeit, Konfliktfähigkeit,
      Kooperationsfähigkeit, Kommunikationsfähigkeit,
      Gender­ und Diversitätskompetenz
    \end{Sozialkompetenzen}
    \begin{Selbstkompetenzen}
      Selbstmanagement, Eigeninitiative,
      Zielorientierung, Stressbewältigung, akademische Integrität
    \end{Selbstkompetenzen}
  \end{Lernergebnisse}

  % specification of contents
  \begin{Inhalt}
    \begin{itemize}
    \item Automatentheorie: deterministische und nichtdeterministische
      Automaten, Kellerautomaten, Turing-Maschinen
    \item Formale Sprachen: reguläre Sprachen, kontextfreie Sprachen,
      Chomsky Hierarchie
    \item Berechenbarkeit und Komplexität: universelle
      Berechenbarkeit, Unentscheidbarkeit, NP-Vollständigkeit
    \item Grundlagen der operationalen und axiomatischen Semantik von
      Programmiersprachen
    \end{itemize}
  \end{Inhalt}

  % prerequisites
  \begin{ErwarteteVorkenntnisse}
    % competencies and knowledge expected from students at the start
    \begin{Intro}
      Grundlegende Kenntnisse der mathematischen Argumentation, der
      Algorithmik und der Modellierung mit Hilfe von Automaten und formalen
      Sprachen.
    \end{Intro}

    % List the modules where students can acquire the prerequisites above.
    \begin{Vormodule}
      Diese Vorausetzungen werden in den Modulen
	\emph{\MODTitel{MathematischesArbeiten}},
        \emph{\MODTitel{AlgorithmenUndDatenstrukturen}} sowie
        \emph{\MODTitel{GrundzuegeDigitalerSysteme}} vermittelt.
    \end{Vormodule}
  \end{ErwarteteVorkenntnisse}

  % Mandatory requirements; usually empty, need to be approved by the curricula commission.
  \begin{VerpflichtendeVoraussetzungen}
    Studieneingangs- und Orientierungsphase.
  \end{VerpflichtendeVoraussetzungen}

  % Teaching methods, examination modes
  \begin{LehrLernformenBeurteilung}
    Die Inhalte werden in einem Vorlesungsteil vorgestellt und in
    begleitenden Übungen von den Studierenden erarbeitet. Die
    Beurteilung erfolgt auf Basis schriftlicher Tests und der
    kontinuierlich in den Übungen erbrachten Leistungen. Der
    Übungsbetrieb und die Tests können computerunterstützt
    durchgeführt werden.
  \end{LehrLernformenBeurteilung}

  % Course(s) implementing the module
  \begin{Lehrveranstaltungen}
    % \LV{Ects}{Sst}{Type}{Title}[UniqueKey]
    \LV{6,0}{4,0}{VU}{Theoretische Informatik}[TheoretischeInformatikVU]
  \end{Lehrveranstaltungen}
\end{Modul}
